
LM shares many similarities with Constraint Handling Rules~
(CHR)~\cite{Betz:2005kx,DBLP:journals/corr/abs-1006-3039}.  CHR is a concurrent
committed-choice constraint language used to write constraint solvers. A CHR
program is a set of rules and a set of constraints (which can be seen as facts).
The constraint store can be see as the database of facts and rules manipulate
the constraint store. Many basic optimizations used in the LM compiler such as
join optimizations and the use of different data structures for indexing facts
were inspired in work done on CHR~\cite{DBLP:journals/corr/cs-PL-0408025}.
Wuille et al.~\cite{42866} have described a CHR implementation in C that follows
some of the ideas presented here and Koninck et al.~\cite{chrp} showed how to
compile CHR programs with dynamic priorities into Prolog. Our work distinguishes
itself from these two works by supporting comprehensions, aggregates and rule
priorities. Compilation of LM programs is
also novel due to the implicit parallelism of rules. Finally, our results are
slightly better than competing CHR systems when compared to C programs.
