
LM shares many similarities~\cite{Betz:2005kx} with Constraint Handling Rules~
(CHR)~\cite{chr}.  CHR is a concurrent committed-choice constraint language used
to write constraint solvers. A CHR program is a set of rules and a set of
constraints.  The constraint store can be seen as a database of facts and rules
manipulate the constraint store. Many basic optimizations used in the LM
compiler such as join optimizations and the use of different data structures for
indexing facts were inspired in work done on
CHR~\cite{DBLP:journals/corr/cs-PL-0408025}.  Wuille et al.~\cite{42866} have
described a CHR to C compiler that follows some of the ideas presented here and
De Koninck et al.~\cite{chrp} showed how to compile CHR programs with dynamic
priorities into Prolog. Our work distinguishes itself from these two works by
supporting a novel combination of comprehensions, aggregates and rule
priorities. Compilation of LM programs is also novel due to the implicit
parallelism of rules, allowing for programs to be
parallelized~\cite{cruz-iclp14}.
